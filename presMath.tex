\documentclass[11pt,a4paper]{beamer}
\usepackage[utf8]{inputenc}
\usepackage[french]{babel}
\usepackage[T1]{fontenc}
\usepackage{amsmath}
\usepackage{amsfonts}
\usepackage{amssymb}
\title{Coq, théorie de la démonstration et CompCert}
\author{Pierre Gimalac, Maxime Flin, Alexandre Moine}
\begin{document}
\maketitle

\begin{frame}
\frametitle{Compcert - Introduction}
\begin{itemize}
\item Compcert est un compilateur certifié pour le langage C\footnote{En fait, CompCert ne compile qu'une grande partie du C, appelée CLight}
\item Compilateur: outil informatique traduisant un programme écrit dans un langage dans un autre langage
\begin{itemize}
	\item Très complexe (surtout avec de l'optimisation)
	\item Donc beaucoup de bugs...
\end{itemize}
\item L'idée est donc de \emph{certifier} mathématiquement ce logiciel si critique
\end{itemize}
\end{frame}

\begin{frame}
\frametitle{COQ à la rescousse}
\end{frame}
\end{document}