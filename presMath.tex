\documentclass[11pt,a4paper]{beamer}

\usetheme{metropolis}

\usepackage[utf8]{inputenc}
\usepackage[french]{babel}
\usepackage[T1]{fontenc}
\usepackage{amsmath}
\usepackage{amsfonts}
\usepackage{amssymb}
\usepackage{bussproofs}

\theoremstyle{plain}
\newtheorem{thm}{Théorème}[section]
\newtheorem{lem}[thm]{Lemme}
\newtheorem{prop}[thm]{Proposition}
\newtheorem*{cor}{Corollaire}

\theoremstyle{definition}
\newtheorem{defn}{Définition}[section]
\newtheorem{conj}{Conjecture}[section]
\newtheorem{exmp}{Exemple}[section]

\theoremstyle{remark}
\newtheorem*{rem}{Remarque}
\newtheorem*{notation}{Notation}

\title{Coq, théorie de la démonstration et CompCert}
\author{Pierre Gimalac, Maxime Flin, Alexandre Moine}

\begin{document}
\maketitle

\section{Logique du premier ordre et d'ordre supérieur}
\begin{frame}
\frametitle{Logique d'ordre supérieur}
Logique du premier ordre permet de quantifier seulement des objets du premier ordres (variables, constantes) mais pas des objets comme des ensembles, des groupes,etc... qui sont d'ordre supérieur.
\newline
\newline
``Toute partie non vide de $\mathbb{N}$ admet un plus petit élément``:
\[ \forall X \left(\neg =(X, \emptyset)\rightarrow\exists p~(X(p)\wedge\forall n~(X(x)\rightarrow \leq(p, n)))\right)\]
\end{frame}

\begin{frame}
\frametitle{Sortes}
\begin{defn}
On choisit $\mathcal{S}_0$ qu'on appellera \textit{ensemble de sortes de base} et qui ne contient pas la sorte particulière $o$. On définit l'ensemble $\mathcal{S}$ des sortes par la grammaire:

\[ \mathcal{S} = o~|~\mathcal{S}_0~|~\mathcal{S}\rightarrow\mathcal{S} \]
\end{defn}

\begin{rem}
\begin{itemize}
\item La sorte $o$ est la sorte des formules
\item On note $\omega$ la sorte des entiers. La sorte $\omega \rightarrow \omega$ désigne la sorte des fonctions de $\mathbb{Z}$ dans lui même.
\item Contrairement à la logique du premier ordre les termes et les termes sont tout deux des formules mais pas de la même sorte.
\end{itemize}
\end{rem}
\end{frame}

\begin{frame}
\frametitle{Un langage d'ordre supérieur}
\begin{defn}
Un \textit{langage d'ordre supérieur} est défini pour un ensemble de sortes $\mathcal{S}$ et est l'ensemble des couples $(c,s)$ où $c$ est un symbole de constante et $s$ une sorte.
\end{defn}

\begin{exmp}
Un langage d'ordre supérieur pour décrire l'arithmétique contiendrait le couple
\[ \left(+, \omega \rightarrow \omega \rightarrow \omega\right) \]
\end{exmp}
\end{frame}

\begin{frame}
\frametitle{Logique d'ordre supérieur}
\begin{defn}
Une \textit{Logique d'ordre supérieur} est un triplet $(\mathcal{S}_0, \mathcal{L}, \mathcal{G})$ où
\begin{itemize}
\item $\mathcal{S}_0$ est l'ensemble des sortes de base
\item $\mathcal{L}$ est le langage d'ordre supérieur sur un ensemble de sortes $\mathcal{S}$ associé
\item $\mathcal{G}$ est un sous ensemble de $\mathcal{S}$. Il détermine l'ensemble des sortes que l'on peut quantifier
\end{itemize}
\end{defn}
\end{frame}

\begin{frame}
\frametitle{$\beta$-équivalence}
\begin{defn}[$\beta$-réduction]
On dit que $e$ se \textbf{$\beta$-réduit} à $e'$ si $e'$ est obtenue à partir de $e$ en remplaçant une sous expression $(x\rightarrow t)(u)$ par l'expression $t[x:=u]$.
\end{defn}

\begin{defn}[$\beta$-équivalence]
Deux expressions sont \textbf{$\beta$-équivalentes} si elles se $\beta$-réduisent, en une ou plusieurs étapes, à la même expression.
\end{defn}

\begin{thm}
Toute expression est égale à une unique expression ne contenant pas de sous expressions de la forme $(x\rightarrow t)(u)$
\end{thm}
\end{frame}

\section{Lambda calcul}
\begin{frame}
\frametitle{Lambda calcul}
\begin{defn}
On construit l'ensemble des $\lambda$-termes $\Lambda$ tel que
\begin{itemize}
\item Pour toute variable $x$, $x\in\Lambda$
\item Pour toute variable $x$ et $M\in\Lambda$, $(\lambda x.M)\in\Lambda$
\item Pour tout $M,N\in\Lambda$, $(M~N)\in\Lambda$
\end{itemize}
\end{defn}

\begin{notation}
\begin{itemize}
\item $M~N$ désigne $(M~N)$
\item $M~N~P$ désigne $((M~N)~P)$
\item $\lambda x.M~N$ désigne $\lambda x.(M~N)$
\item $\lambda xyz.M$ désigne $\lambda x.\lambda y. \lambda z. M$
\end{itemize}
\end{notation}
\end{frame}

\begin{frame}
\frametitle{$\alpha$-conversion, $\beta$-réduction}
\begin{defn}[$\alpha$-conversion]
Une $\alpha$-conversion est un changement de nom pour une abstraction. On change le nom de la variable partout où elle est correspond à cette abstraction et pour un nom de variable qui n'est pas abstrait.
\end{defn}

\begin{defn}[$\beta$-réduction]
La $\beta$-réduction traduit l'idée d'appliquer la fonction décrite par une abstraction. Ainsi, $(\lambda x.E)~E'\rightarrow_\beta E[x:=E']$ 
\end{defn}

\begin{defn}[$\beta$-équivalence]
On dit que deux $\lambda$-termes sont équivalents quand ils se $\beta$-réduisent au même $\lambda$-terme.
\end{defn}
\end{frame}

\section{Théorie de la démonstration}
\begin{frame}
\frametitle{La notion de séquent}
\begin{defn}
Un \textbf{séquent} est un couple noté $\Gamma \vdash F$ où
\begin{itemize}
	\item $\Gamma$ est un ensemble fini de formules. On l'appelle contexte
	\item $F$ est une formule. On l'appelle conclusion
\end{itemize}
\end{defn}

\begin{rem}
Les formules de $\Gamma$ et $F$ ne sont pas nécessairement closes.
\end{rem}
\end{frame}

\begin{frame}
\frametitle{La notion de séquent}
\begin{defn}
Un séquent $\Gamma \vdash F$ est \textbf{démontrable} s'il existe une démonstration ayant ses hypothèses dans $\Gamma$ et $F$ pour conclusion.
\end{defn}

\begin{rem}
\begin{itemize}
	\item Quand on écrit $\Gamma \vdash F$ on considère que ce séquent est démontrable.
	\item Quand un séquent n'est pas prouvable, on écrit $\Gamma \nvdash F$.
	\item Si $\vdash F$ on dit que $F$ est une tautologie.
	\item Si $\nvdash F$ on dit que $F$ est une antilogie.
\end{itemize}
\end{rem}
\end{frame}

\begin{frame}
\frametitle{Définition de démonstration formelle}
\begin{defn}
Une \textbf{démonstration} est un arbre ayant pour feuille des séquents.

Tous les arbres de ce type ne sont bien sûr pas de démonstrations. On les construit l'ensemble des démonstration formelles par induction comme suit.
\end{defn}
\end{frame}

\begin{frame}
\frametitle{Règles d'induction}
\begin{defn}[L'axiome]
pour toute théorie $\Gamma$ et formule $A$
\begin{prooftree}
\AxiomC{}
\RightLabel{ax}
\UnaryInfC{$\Gamma,A\vdash A$}
\end{prooftree}
\end{defn}

\begin{defn}[L'affaiblissement]
Pour toute théorie $\Gamma$ et formule $A$
\begin{prooftree}
\AxiomC{$\Gamma\vdash A$}
\RightLabel{aff}
\UnaryInfC{$\Gamma,B\vdash A$}
\end{prooftree}
\end{defn}
\end{frame}

\begin{frame}
Pour toute théorie $\Gamma$ et formules $A, B$
\frametitle{Règles d'induction}
\begin{defn}[Introduction de l'implication]
\begin{prooftree}
\AxiomC{$\Gamma, A\vdash B$}
\RightLabel{$\rightarrow_i$}
\UnaryInfC{$\Gamma\vdash A \rightarrow B$}
\end{prooftree}
\end{defn}

\begin{defn}[Introduction de la conjonction]
\begin{prooftree}
\AxiomC{$\Gamma\vdash A$}
\AxiomC{$\Gamma \vdash B$}
\RightLabel{$\wedge_i$}
\BinaryInfC{$\Gamma\vdash A \rightarrow B$}
\end{prooftree}
\end{defn}

\begin{defn}[Introduction de la disjonction]
\begin{prooftree}
\AxiomC{$\Gamma\vdash A$}
\RightLabel{$\vee_i$}
\UnaryInfC{$\Gamma\vdash A \vee B$}
\end{prooftree}
\end{defn}
\end{frame}

\begin{frame}
\frametitle{Règles d'induction}
\begin{defn}[Élimination de l'implication]
\begin{prooftree}
\AxiomC{$\Gamma\vdash A \rightarrow B$}
\AxiomC{$\Gamma\vdash A$}
\RightLabel{$\rightarrow_e$}
\BinaryInfC{$\Gamma\vdash B$}
\end{prooftree}
\end{defn}

\begin{defn}[Élimination de la conjonction]
\begin{prooftree}
\AxiomC{$\Gamma\vdash A\wedge B$}
\RightLabel{$\wedge_e$}
\UnaryInfC{$\Gamma\vdash A$}
\end{prooftree}
\end{defn}

\begin{defn}[Élimination de la disjonction]
Pour toute théorie $\Gamma$ et formules $A, B, C$
\begin{prooftree}
\AxiomC{$\Gamma\vdash A\vee B$}
\AxiomC{$\Gamma, A \vdash C$}
\AxiomC{$\Gamma, B \vdash C$}
\RightLabel{$\vee_i$}
\TrinaryInfC{$\Gamma\vdash C$}
\end{prooftree}
\end{defn}
\end{frame}

\begin{frame}
\frametitle{Règles d'induction}
\begin{defn}[Élimination par l'absurde]
\begin{prooftree}
\AxiomC{$\Gamma,\neg A\vdash \perp$}
\RightLabel{$\perp_e$}
\UnaryInfC{$\Gamma\vdash A$}
\end{prooftree}
\end{defn}
\end{frame}

\begin{frame}
\frametitle{Règles d'induction}
\begin{defn}[Introduction du quantificateur universel]
Si $x$ n'est libre dans aucune formule de $\Gamma$
\begin{prooftree}
\AxiomC{$\Gamma\vdash A$}
\RightLabel{$\forall_i$}
\UnaryInfC{$\Gamma\vdash\forall x A$}
\end{prooftree}
\end{defn}

\begin{defn}[Introduction du quantificateur existentiel]
\begin{prooftree}
\AxiomC{$\Gamma\vdash A[x:=t]$}
\RightLabel{$\exists_i$}
\UnaryInfC{$\Gamma\vdash\exists x~A$}
\end{prooftree}
\end{defn}
\end{frame}

\begin{frame}
\frametitle{Règles d'induction}
\begin{defn}[Élimination du quantificateur universel]
\begin{prooftree}
\AxiomC{$\Gamma\vdash\forall x~A$}
\RightLabel{$\forall_e$}
\UnaryInfC{$\Gamma\vdash A[x:=t]$}
\end{prooftree}
\end{defn}

\begin{defn}[Introduction du quantificateur existentiel]
Si $x$ n'est libre ni dans $C$ ni dans aucune formule de $\Gamma$
\begin{prooftree}
\AxiomC{$\Gamma\vdash \exists x~A$}
\AxiomC{$\Gamma,A\vdash C$}
\RightLabel{$\exists_e$}
\BinaryInfC{$\Gamma\vdash C$}
\end{prooftree}
\end{defn}
\end{frame}

\begin{frame}
\begin{defn}[Introduction de l'égalité]
\begin{prooftree}
\AxiomC{}
\RightLabel{$=_i$}
\UnaryInfC{$\Gamma\vdash t=t$}
\end{prooftree}
\end{defn}

\begin{defn}[Élimination de l'égalité]
\begin{prooftree}
\AxiomC{$\Gamma\vdash A[x:=t]$}
\AxiomC{$\Gamma\vdash t=u$}
\RightLabel{$=_e$}
\BinaryInfC{$\Gamma\vdash A[x:=u]$}
\end{prooftree}
\end{defn}
\end{frame}

\begin{frame}
\frametitle{Exemple de démonstration formelle}
On considère $\Gamma = \left\{\forall x~H(x)\rightarrow M(x), H(s)\right\}$
\begin{prooftree}
\AxiomC{}
\RightLabel{ax}
\UnaryInfC{$\Gamma\vdash \forall x~H(x)\rightarrow M(x)$}
\RightLabel{$\forall_e$}
\UnaryInfC{$\Gamma\vdash H(s)\rightarrow M(s)$}

\AxiomC{}
\RightLabel{ax}
\UnaryInfC{$\Gamma\vdash H(s)$}
\RightLabel{$\rightarrow_e$}
\BinaryInfC{$\Gamma\vdash M(s)$}
\end{prooftree}

C'est un syllogisme !

\end{frame}


\section{Transformation de Curry-Howard}

\begin{frame}
\frametitle{Une correspondance entre la logique et la programmation}
\begin{center}
\begin{tabular}{|c|c|}
\hline 
Logique & Programmation \\ 
\hline 
\hline
Hilbert & Logique combinatoire \\ 
\hline 
Déduction naturelle & $\lambda$-calcul \\ 
\hline 
hypothèse & variable libre \\ 
\hline 
élimination de l'implication & application \\ 
\hline 
introduction de l'implication & abstraction \\
\hline
\end{tabular} 
\end{center}
\end{frame}

\begin{frame}
\frametitle{Une correspondance entre la logique et la programmation}

\begin{prooftree}
\AxiomC{}
\RightLabel{ax}
\UnaryInfC{$\Gamma_1,\alpha\vdash\alpha$}
\end{prooftree}

\begin{prooftree}
\AxiomC{}
\UnaryInfC{$\Gamma,x:\alpha\vdash x:\alpha$}
\end{prooftree}
\end{frame}

\begin{frame}
\frametitle{Une correspondance entre la logique et la programmation}

\begin{prooftree}
\AxiomC{$\Gamma,\alpha\vdash\beta$}
\RightLabel{$\rightarrow_i$}
\UnaryInfC{$\Gamma\vdash\alpha\rightarrow\beta$}
\end{prooftree}

\begin{prooftree}
\AxiomC{$\Gamma,x:\alpha\vdash t:\beta$}
\UnaryInfC{$\Gamma\vdash \lambda .x:\alpha\rightarrow\beta$}
\end{prooftree}
\end{frame}

\begin{frame}
\frametitle{Une correspondance entre la logique et la programmation}

\begin{prooftree}
\AxiomC{$\Gamma\vdash\alpha\rightarrow\beta$}
\AxiomC{$\Gamma\vdash\alpha$}
\RightLabel{$\rightarrow_e$}
\BinaryInfC{$\Gamma\vdash\beta$}
\end{prooftree}

\begin{prooftree}
\AxiomC{$\Gamma\vdash t:\alpha\rightarrow\beta$}
\AxiomC{$\Gamma\vdash x:\alpha$}
\BinaryInfC{$\Gamma\vdash t x:\beta$}
\end{prooftree}
\end{frame}


\begin{frame}
\frametitle{Compcert - Introduction}
\begin{itemize}
\item Compcert est un compilateur certifié pour le langage C\footnote{En fait, CompCert ne compile qu'une grande partie du C, appelée CLight}
\item Compilateur: outil informatique traduisant un programme écrit dans un langage dans un autre langage
\begin{itemize}
	\item Très complexe (surtout avec de l'optimisation)
	\item Donc beaucoup de bugs...
\end{itemize}
\item L'idée est donc de \emph{certifier} mathématiquement ce logiciel si critique
\end{itemize}
\end{frame}

\begin{frame}
\frametitle{COQ à la rescousse}
COQ a donc été utilisé dans CompCert de plusieurs manières
\begin{itemize}
	\item Comme langage de programmation
	\item Comme langage de vérification
	\item Comme langage exportable vers Caml
\end{itemize}
\end{frame}
\end{document}