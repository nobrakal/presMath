\documentclass[11pt,a4paper]{beamer}

\usetheme{metropolis}

\usepackage[utf8]{inputenc}
\usepackage[french]{babel}
\usepackage[T1]{fontenc}
\usepackage{amsmath}
\usepackage{amsfonts}
\usepackage{amssymb}
\usepackage{bussproofs}
\usepackage{listings}
\usepackage{hyperref}

\theoremstyle{plain}
\newtheorem{thm}{Théorème}[section]
\newtheorem{lem}[thm]{Lemme}
\newtheorem{prop}[thm]{Proposition}
\newtheorem*{cor}{Corollaire}

\theoremstyle{definition}
\newtheorem{defn}{Définition}[section]
\newtheorem{conj}{Conjecture}[section]
\newtheorem{exmp}{Exemple}[section]

\theoremstyle{remark}
\newtheorem*{rem}{Remarque}
\newtheorem*{notation}{Notation}

\title{Coq, théorie de la démonstration et CompCert}
\author{Pierre Gimalac, Maxime Flin, Alexandre Moine}

\begin{document}
\maketitle

\begin{frame}{Sommaire}
\tableofcontents
\end{frame}

\section{Compcert}

\begin{frame}{Introduction}
Pourquoi s'amuser à \textit{certifier} un compilateur ?
\begin{itemize}
	\item Ils peuvent introduire des bugs.
	\item Ils sont \textbf{indispensables}!
\end{itemize}
\end{frame}

\begin{frame}{COQ à la rescousse}
Coq est:
\begin{itemize}
\item Un langage de programmation fonctionnelle pur.
\item Un assistant de preuves.
\item Un langage exportable vers OCaml.
\end{itemize}
Mais comment ça marche ? Sur quelles bases mathématiques ?
\end{frame}

\section{Logique du premier ordre et d'ordre supérieur}
\begin{frame}
\frametitle{Logique d'ordre supérieur}
$$\forall x\forall y (=(+(x,y),+(y,x))$$

``Toute partie non vide de $\mathbb{N}$ admet un plus petit élément``:
\[ \forall X \left(\neg =(X, \emptyset)\rightarrow\exists p~(X(p)\wedge\forall n~(X(x)\rightarrow \leq(p, n)))\right)\]
\end{frame}

\begin{frame}
\frametitle{Sortes}
\begin{defn}
On choisit $\mathcal{S}_0$ qu'on appellera \textit{ensemble de sortes de base} et qui ne contient pas la sorte particulière $o$. On définit l'ensemble $\mathcal{S}$ des sortes par la grammaire:

\[ \mathcal{S} = o~|~\mathcal{S}_0~|~\mathcal{S}\rightarrow\mathcal{S} \]
\end{defn}

\begin{exmp}
Un langage d'ordre supérieur pour décrire l'arithmétique contiendrait le couple
\[ \left(+, \omega \rightarrow \omega \rightarrow \omega\right) \]
\end{exmp}
\end{frame}

\begin{frame}
\frametitle{$\beta$-équivalence}
\begin{defn}[$\beta$-réduction]
On dit que $e$ se \textbf{$\beta$-réduit} à $e'$ si $e'$ est obtenue à partir de $e$ en remplaçant une sous expression $(x\rightarrow t)(u)$ par l'expression $t[x:=u]$.
\end{defn}

\begin{defn}[$\beta$-équivalence]
Deux expressions sont \textbf{$\beta$-équivalentes} si elles se $\beta$-réduisent, en une ou plusieurs étapes, à la même expression.
\end{defn}

\begin{thm}
Toute expression est égale à une unique expression ne contenant pas de sous expressions de la forme $(x\rightarrow t)(u)$
\end{thm}
\end{frame}

\section{Lambda calcul}
\begin{frame}
\frametitle{Lambda calcul}
\begin{defn}
On construit l'ensemble des $\lambda$-termes $\Lambda$ tel que
\begin{itemize}
\item Pour toute variable $x$, $x\in\Lambda$
\item Pour toute variable $x$ et $M\in\Lambda$, $(\lambda x.M)\in\Lambda$
\item Pour tout $M,N\in\Lambda$, $(M~N)\in\Lambda$
\end{itemize}
\end{defn}

\begin{notation}
\begin{itemize}
\item $M~N$ désigne $(M~N)$
\item $M~N~P$ désigne $((M~N)~P)$
\item $\lambda x.M~N$ désigne $\lambda x.(M~N)$
\item $\lambda xyz.M$ désigne $\lambda x.\lambda y. \lambda z. M$
\end{itemize}
\end{notation}
\end{frame}

\begin{frame}
\frametitle{$\alpha$-conversion, $\beta$-réduction}
\begin{defn}[$\alpha$-conversion]
Une $\alpha$-conversion est un changement de nom pour une abstraction. On change le nom de la variable partout où elle est correspond à cette abstraction et pour un nom de variable qui n'est pas abstrait.
\end{defn}

\begin{defn}[$\beta$-réduction]
La $\beta$-réduction traduit l'idée d'appliquer la fonction décrite par une abstraction. Ainsi, $(\lambda x.E)~E'\rightarrow_\beta E[x:=E']$ 
\end{defn}

\begin{defn}[$\beta$-équivalence]
On dit que deux $\lambda$-termes sont équivalents quand ils se $\beta$-réduisent au même $\lambda$-terme.
\end{defn}
\end{frame}

\begin{frame}{$\lambda$-calcul simplement typé - Le problème}
$$ Y = \lambda f . (\lambda x. f (xx)) (\lambda x.f (xx))$$
$$ \Omega = (\lambda x \ . \ xx) (\lambda x \ . \ xx) $$

Certains termes ne se $\beta$-réduisent pas.
\end{frame}

\begin{frame}{$\lambda$-calcul simplement typé - La solution (à la Curry)}
\begin{defn}[Environnement]
Un environnement $\Gamma$ est une suite de variables associées à un type: $$\Gamma = x_0:\sigma_0;...;x_n:\sigma_n$$
\end{defn}
\begin{defn}[jugement]
Un jugement d'un lambda terme $M$ est l'affirmation de son type dans un environnement $\Gamma$:
$$ \Gamma \vdash M : \sigma $$
\end{defn}
\end{frame}

\begin{frame}{Les règles de typage}
\begin{itemize}
	\item  $ \Gamma = x:\sigma, \quad \Gamma \vdash x : \sigma $
	\item  $ \Gamma = x:\sigma;M:\tau, \quad \Gamma \vdash \lambda x.M : \sigma \to \tau $
	\item  $ \Gamma = M:\sigma \to \tau; N: \sigma, \quad \Gamma \vdash MN : \tau $
\end{itemize}
On voit que ce système permet de préserver une cohérence des $\lambda$-termes dans un certains environnement
\end{frame}

\begin{frame}{Quelques résultats}
\begin{prop}
La $\beta$-réduction préserve les types
\end{prop}

\begin{prop}[Normalisation forte]
Toute $\beta$-réduction d'un $\lambda$-terme correctement typé termine
\end{prop}
\end{frame}

\section{Théorie de la démonstration}
\begin{frame}
\frametitle{La notion de séquent}
\begin{defn}
Un \textbf{séquent} est un couple noté $\Gamma \vdash F$ où
\begin{itemize}
	\item $\Gamma$ est un ensemble fini de formules. On l'appelle contexte
	\item $F$ est une formule. On l'appelle conclusion
\end{itemize}
\end{defn}

\begin{rem}
Les formules de $\Gamma$ et $F$ ne sont pas nécessairement closes.
\end{rem}
\end{frame}

\begin{frame}
\frametitle{La notion de séquent}
\begin{defn}
Un séquent $\Gamma \vdash F$ est \textbf{démontrable} s'il existe une démonstration ayant ses hypothèses dans $\Gamma$ et $F$ pour conclusion.
\end{defn}

\begin{rem}
\begin{itemize}
	\item Quand on écrit $\Gamma \vdash F$ on considère que ce séquent est démontrable.
	\item Quand un séquent n'est pas prouvable, on écrit $\Gamma \nvdash F$.
	\item Si $\vdash F$ on dit que $F$ est une tautologie.
	\item Si $\nvdash F$ on dit que $F$ est une antilogie.
\end{itemize}
\end{rem}
\end{frame}

\begin{frame}
\frametitle{Définition de démonstration formelle}
\begin{defn}
Une \textbf{démonstration} est un arbre étiqueté par des séquents.

Tous les arbres de ce type ne sont bien sûr pas de démonstrations. On les construit l'ensemble des démonstration formelles par induction en définissant des règles d'induction.
\end{defn}
\end{frame}

\begin{frame}
\frametitle{Règles d'induction}
\begin{defn}[L'axiome]
pour toute théorie $\Gamma$ et formule $A$
\begin{prooftree}
\AxiomC{}
\RightLabel{ax}
\UnaryInfC{$\Gamma,A\vdash A$}
\end{prooftree}
\end{defn}

\begin{defn}[L'affaiblissement]
Pour toute théorie $\Gamma$ et formule $A$
\begin{prooftree}
\AxiomC{$\Gamma\vdash A$}
\RightLabel{aff}
\UnaryInfC{$\Gamma,B\vdash A$}
\end{prooftree}
\end{defn}
\end{frame}

\begin{frame}
Pour toute théorie $\Gamma$ et formules $A, B$
\frametitle{Règles d'induction}
\begin{defn}[Introduction de l'implication]
\begin{prooftree}
\AxiomC{$\Gamma, A\vdash B$}
\RightLabel{$\rightarrow_i$}
\UnaryInfC{$\Gamma\vdash A \rightarrow B$}
\end{prooftree}
\end{defn}

\begin{defn}[Introduction de la conjonction]
\begin{prooftree}
\AxiomC{$\Gamma\vdash A$}
\AxiomC{$\Gamma \vdash B$}
\RightLabel{$\wedge_i$}
\BinaryInfC{$\Gamma\vdash A \rightarrow B$}
\end{prooftree}
\end{defn}

\begin{defn}[Introduction de la disjonction]
\begin{prooftree}
\AxiomC{$\Gamma\vdash A$}
\RightLabel{$\vee_i$}
\UnaryInfC{$\Gamma\vdash A \vee B$}
\end{prooftree}
\end{defn}
\end{frame}

\begin{frame}
\frametitle{Règles d'induction}
\begin{defn}[Élimination de l'implication]
\begin{prooftree}
\AxiomC{$\Gamma\vdash A \rightarrow B$}
\AxiomC{$\Gamma\vdash A$}
\RightLabel{$\rightarrow_e$}
\BinaryInfC{$\Gamma\vdash B$}
\end{prooftree}
\end{defn}

\begin{defn}[Élimination de la conjonction]
\begin{prooftree}
\AxiomC{$\Gamma\vdash A\wedge B$}
\RightLabel{$\wedge_e$}
\UnaryInfC{$\Gamma\vdash A$}
\end{prooftree}
\end{defn}

\begin{defn}[Élimination de la disjonction]
Pour toute théorie $\Gamma$ et formules $A, B, C$
\begin{prooftree}
\AxiomC{$\Gamma\vdash A\vee B$}
\AxiomC{$\Gamma, A \vdash C$}
\AxiomC{$\Gamma, B \vdash C$}
\RightLabel{$\vee_i$}
\TrinaryInfC{$\Gamma\vdash C$}
\end{prooftree}
\end{defn}
\end{frame}

\begin{frame}
\frametitle{Exemple de démonstration formelle}
On considère $\Gamma = \left\{\forall x~H(x)\rightarrow M(x), H(s)\right\}$
\begin{prooftree}
\AxiomC{}
\RightLabel{ax}
\UnaryInfC{$\Gamma\vdash \forall x~H(x)\rightarrow M(x)$}
\RightLabel{$\forall_e$}
\UnaryInfC{$\Gamma\vdash H(s)\rightarrow M(s)$}

\AxiomC{}
\RightLabel{ax}
\UnaryInfC{$\Gamma\vdash H(s)$}
\RightLabel{$\rightarrow_e$}
\BinaryInfC{$\Gamma\vdash M(s)$}
\end{prooftree}

C'est un syllogisme !

\end{frame}


\section{Transformation de Curry-Howard}

\begin{frame}
\frametitle{Une correspondance entre la logique et la programmation}
\begin{center}
\begin{tabular}{|c|c|}
\hline 
Logique & Programmation \\ 
\hline 
\hline
Hilbert & Logique combinatoire \\ 
\hline 
Déduction naturelle & $\lambda$-calcul \\ 
\hline 
hypothèse & variable libre \\ 
\hline 
élimination de l'implication & application \\ 
\hline 
introduction de l'implication & abstraction \\
\hline
\end{tabular} 
\end{center}
\end{frame}

\begin{frame}
\frametitle{Une correspondance entre la logique et la programmation}

\begin{prooftree}
\AxiomC{}
\RightLabel{ax}
\UnaryInfC{$\Gamma_1,\alpha\vdash\alpha$}
\end{prooftree}
$$  \Updownarrow $$
\begin{prooftree}
\AxiomC{}
\UnaryInfC{$\Gamma,x:\alpha\vdash x:\alpha$}
\end{prooftree}
\end{frame}

\begin{frame}
\frametitle{Une correspondance entre la logique et la programmation}

\begin{prooftree}
\AxiomC{$\Gamma,\alpha\vdash\beta$}
\RightLabel{$\rightarrow_i$}
\UnaryInfC{$\Gamma\vdash\alpha\rightarrow\beta$}
\end{prooftree}
$$  \Updownarrow $$
\begin{prooftree}
\AxiomC{$\Gamma,x:\alpha\vdash t:\beta$}
\UnaryInfC{$\Gamma\vdash \lambda .x:\alpha\rightarrow\beta$}
\end{prooftree}
\end{frame}

\begin{frame}
\frametitle{Une correspondance entre la logique et la programmation}

\begin{prooftree}
\AxiomC{$\Gamma\vdash\alpha\rightarrow\beta$}
\AxiomC{$\Gamma\vdash\alpha$}
\RightLabel{$\rightarrow_e$}
\BinaryInfC{$\Gamma\vdash\beta$}
\end{prooftree}
$$  \Updownarrow $$
\begin{prooftree}
\AxiomC{$\Gamma\vdash t:\alpha\rightarrow\beta$}
\AxiomC{$\Gamma\vdash x:\alpha$}
\BinaryInfC{$\Gamma\vdash t x:\beta$}
\end{prooftree}
\end{frame}

\section{Coq}

\begin{frame}{Introduction}
Coq est un \emph{assistant de preuve} développé par l'INRIA. Grâce à une sémantique précise (le langage \emph{Gallina}) il permet l'automatisation de la vérification de preuves mathématiques via l'isomorphisme de Curry-Howard.\\
Un de ses grands avantages est que Gallina est un langage fonctionnel, et donc qu'il permet aussi d'écrire des logiciels.
\end{frame}

\begin{frame}{Manque de Turing-complétude}
\emph{Gallina} est un langage fonctionnel typé:
\begin{itemize}
	\item Tout programme est donc fortement normalisant
	\item Donc tout programme termine
	\item Donc \emph{Gallina} n'est pas Turing-complet\footnote{Un langage Turing-complet est un langage ayant autant d'expressivité qu'une machine de Turing}
\end{itemize}

\end{frame}

\begin{frame}{Utilisations}
	\begin{itemize}
		\item La preuve du théorème des 4 couleurs (2005)
		\item La preuve du théorème de Feit et Thompson (2012)
		\item Compcert...
	\end{itemize}
\end{frame}

\begin{frame}[fragile]{Preuve de $A\rightarrow A$}
\begin{lstlisting}[frame=single]
Theorem a_implies_a: forall A, A -> A.
Proof.
intros A.
intros H.
exact H.
Qed.
\end{lstlisting}

Donne la fonction identité
\begin{lstlisting}[frame=single][language=Caml]
(* val a_implies_a: 'a1 -> 'a1 *)
let a_implies_a h = h
\end{lstlisting}
\end{frame}

\begin{frame}[fragile]{Preuve de $A\rightarrow A$}
\begin{lstlisting}[frame=single]
Theorem a_implies_a: forall A, A -> A.
Proof.
intros A.
exact (fun (t:A) => t).
Qed.
\end{lstlisting}
\end{frame}

\section{Retour à Compcert}

\begin{frame}{Comment c'est fait ?}
	\begin{itemize}
		\item Transforme le fichier en arbre de syntaxe abstrait (non-vérifié) de CMinor
		\item Transforme cet arbre de syntaxe abstrait de CMinor en arbre de syntaxe PowerPC
		\item On écrit le code assembleur correspondant
	\end{itemize}
Le tout est écrit \emph{ET} prouvé en Coq $\implies$ une preuve du simple compilateur, puisque la preuve est le programme.
\end{frame}

\begin{frame}{Les difficultés}
	\begin{itemize}
		\item Il faut spécifier rigoureusement les langages utilisés, en coq
		\item Les compilateurs sont traditionnellement écrit de manière impérative, il faut raisonner de manière fonctionnelle.
	\end{itemize}
\end{frame}

\begin{frame}{Comment certifier un compilateur ?}
Soit $S$ le programme source et $C$ le programme compilé 
	\begin{itemize}
		\item On veut que $C$ ai le même comportement que $S$
		\item C'est une condition trop forte (Les compilateurs choisissent un ordre de calcul des expressions arithmétique parmi tous ceux possibles)
		\item On va donc se concentrer sur: Si $S$ ne s'évalue pas en un comportement mauvais, alors $C$ a le même comportement que $S$.
	\end{itemize}
\end{frame}

\section*{Sources}
\begin{frame}
\begin{itemize}
	\item \url{https://coq.inria.fr/}
	\item \url{http://compcert.inria.fr/}
	\item \url{http://perso.ens-lyon.fr/pierre.lescanne/ENSEIGNEMENT/PROG2/06-07/lambda_types.pdf}
\end{itemize}
\end{frame}
\end{document}
